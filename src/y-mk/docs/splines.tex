\documentclass[aps,twocolumn]{revtex4}
\usepackage{graphicx}
\usepackage{amssymb,amsfonts,amsmath,amsthm}
\usepackage{chemarr}
\usepackage{bm}
\usepackage{pslatex}
\usepackage{mathptmx}
\usepackage{xfrac}

%% concentration notations
\newcommand{\mymat}[1]{\boldsymbol{#1}}
\newcommand{\mytrn}[1]{{#1}^{\mathsf{T}}}
\newcommand{\myvec}[1]{\overrightarrow{#1}}
\newcommand{\mygrad}{\vec{\nabla}}
\newcommand{\myhess}{\mathcal{H}}


\begin{document}
\title{Cubic Splines}
\maketitle

Let us assume that we have a signal based on 
$$x_1<\ldots < x_i < \ldots<x_N$$
with values 
$$y_1,\ldots,y_i,\ldots,y_n.$$
The linear interpolation between $x_j$ and $x_{j+1}$ writes
$$
	y = A y_{j} + B y_{j+1}
$$
with
$$
	A = \dfrac{x_{j+1}-x}{x_{j+1}-x_{j}},\;\;
	B = (1-A) = \dfrac{x-x_{j}}{x_{j+1}-x_{j}}.
$$
Let us assume that we know the second derivatives
$$
	y_1'',\ldots,y_i'',\ldots,y_n''.
$$
The we have
$$
	y = A y_{j} + B y_{j+1} + C y_j'' + D y_{j+1}''
$$
with
$$
	C = \dfrac{1}{6}\left(A^3-A\right)\left(x_{j+1}-x_{j}\right)^2,\;\;D=\dfrac{1}{6}\left(B^3-B\right)\left(x_{j+1}-x_{j}\right)^2.
$$
The cubic continuous condition is met when the first derivative is continuous
$$
	y' = \dfrac{y_{j+1}-y_{j}}{x_{j+1}-x_{j}} - \dfrac{3A^2-1}{6}\left(x_{j+1}-x_{j}\right)y_j'' +
	\dfrac{3B^2-1}{6}\left(x_{j+1}-x_{j}\right)y_{j+1}''
$$
\end{document}
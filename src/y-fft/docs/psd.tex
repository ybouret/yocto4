\documentclass[aps,twocolumn]{revtex4}
\usepackage{graphicx}
\usepackage{amssymb,amsfonts,amsmath,amsthm}
\usepackage{chemarr}
\usepackage{bm}
\usepackage{pslatex}
\usepackage{mathptmx}
\usepackage{xfrac}

%% concentration notations
\newcommand{\mymat}[1]{\boldsymbol{#1}}
\newcommand{\mytrn}[1]{{#1}^{\mathsf{T}}}
\newcommand{\myvec}[1]{\overrightarrow{#1}}
\newcommand{\mygrad}{\vec{\nabla}}
\newcommand{\myhess}{\mathcal{H}}


\begin{document}
\title{Thoughts for Power Spectrum Estimation}
\maketitle

\section{Modifying signal}
Let us assume that we have $f(t)$ on an interval $T$.
We shall use the function
\begin{equation}
	g(t) = f(t) + a_1 t + a_0
\end{equation}
that corrects the drift of $f$.
For this, we want to minimize
\begin{equation}
	J = \int_0^T g(t)^2 \, dt.
\end{equation}
We want to set
\begin{equation}
0 = \dfrac{\partial J}{\partial a_0} = \int_0^T g(t)\dfrac{\partial g}{\partial a_0} \, dt = \int_0^T g(t) dt
\end{equation}
meaning that we want $g$ to have a zero integral over $T$, which is 
expected from an harmonic signal.
We also want to set
\begin{equation}
0 = \dfrac{\partial J}{\partial a_1} = \int_0^T g(t)\dfrac{\partial g}{\partial a_1} \, dt = \int_0^T t g(t) dt
\end{equation}
which shall remove the constant drift part of $g$.
With 
\begin{equation}
	\left\lbrace
	\begin{array}{rcl}
	\sigma_1 & = & \sum_i f_i \\
	\sigma_2 & = & \sum_i \left(i+\frac{1}{2}\right) f_i\\
	\end{array}
	\right.
\end{equation}
we obtain
\begin{equation}
	\left\lbrace
	\begin{array}{rcl}
	a_0 & = & 6 \dfrac{\sigma_2}{M^2} - 4 \dfrac{\sigma_1}{M}\\
	\\
	a_1 & = & 6 \dfrac{\sigma_1}{M^2} - 12 \dfrac{\sigma_2}{M^3}\\
	\end{array}
	\right.
\end{equation}

\end{document}
\documentclass[aps,twocolumn]{revtex4}
\usepackage{graphicx}
\usepackage{amssymb,amsfonts,amsmath}
\usepackage{chemarr}
\usepackage{bm}
\usepackage{pslatex}
%\usepackage{mathptmx}
\usepackage{xfrac}

%% concentration notations
\newcommand{\mymat}[1]{\boldsymbol{#1}}
%\newcommand{\mytrn}[1]{{#1}^{\mathrm{t}}}
\newcommand{\mytrn}[1]{{\!\!~^{\mathsf{t}}{#1}}}
\newcommand{\myvec}[1]{\overrightarrow{#1}}
\newcommand{\half}{\sfrac{1}{2}}
\newcommand{\q}{\vec{q}}
\newcommand{\dq}{\dot{\q}}
\newcommand{\ddq}{\ddot{\q}}
\newcommand{\C}{\vec{C}}
\newcommand{\J}{\mymat{J}}
\newcommand{\dJ}{\dot{\J}}
\newcommand{\tJ}{\mytrn{\J}}
\newcommand{\G}{\vec{G}}
\newcommand{\W}{\mymat{W}}
\newcommand{\A}{\mymat{A}}
\newcommand{\JW}{\hat{\mymat{J}}}
\newcommand{\tJW}{\mytrn{\JW}}

\begin{document}
\title{Dynamics}

\section{Notations}
Let us have a system with $N$ particles, so that we have a $3N$-dimensional
vector $\q$ of positions and a vector $\dq$ of velocities.
We assume that we can compute the force field $\vec{F}$ such that
\begin{equation}
	\label{eq:newton}
	\ddq = \mymat{W} \vec{F}
\end{equation}
where $\mymat{W}$ is the tensorized mass inverse matrix.

\section{Newtonian Dynamics}

\subsection{Problem}
At step $n$, we know $\q_n$, $\dq_n$ and $\ddq_n = \mymat{W} \vec{F}_n$.
How do we compute $\q_{n+1}$ and $\dq_{n+1}$after a time step $\tau$ ?

\subsection{Velocity Verlet Derivation for Holonomic Forces}
We update $\q_n$ by Taylor expansion during $\tau$ with
\begin{equation}
	\label{eq:forward}
	\q_{n+1} = \q_n + \tau \dq_n + \frac{1}{2}\tau^2 \ddq_n.
\end{equation}
Since we can evaluate $\vec{F}_{n+1}$ and $\ddq_{n+1}=\mymat{W}\vec{F}_{n+1}$, the backward Taylor expansion
gives
\begin{equation}
	\label{eq:backward}
	\q_n = \q_{n+1} - \tau \dq_{n+1} + \frac{1}{2}\tau^2 \ddq_{n+1}.
\end{equation}
The sum of Eq. \eqref{eq:forward} and Eq. \eqref{eq:backward} gives 
\begin{equation}
	\label{eq:vv}
	\dq_{n+1} = \dq_{n} + \dfrac{1}{2}\mymat{W}\left\lbrack\vec{F}_n + \vec{F}_{n+1}\right\rbrack
\end{equation}

\subsection{Velocity Verlet for Generic Forces}
We use the following sequence.
Given $\q_n$ and $\dq_n$, we compute $\vec{F}_n=\vec{F}(\q_n,\dq_n)$.
Then for each time step we perform the following sequence.
\begin{enumerate}
	\item Evaluate \begin{equation} 
	\dq_{n+\half} = \dq_n + \frac{1}{2}\tau \mymat{W}\vec{F}_n
	\end{equation}
		\item Advance \begin{equation}
	\q_{n+1}  = \q_n + \tau \dq_{n+\half}.
	\end{equation} 
	The potential energy is already available.
	\item Evaluate $\vec{F}_{n+\half}=\vec{F}(\q_{n+1},\dq_{n+\half})$
	\item Update
	\begin{equation}
		\dq_{n+1} = \dq_{n+\half} + \frac{1}{2} \tau \W \vec{F}_{n+\half}.
	\end{equation}
	The kinetic energy if available.
	\item Update
	\begin{equation}
		\vec{F}_{n+1}=\vec{F}(\q_{n+1},\dq_{n+1})
	\end{equation}
\end{enumerate}

\section{Holonomic Constraints}
\subsection{Definitions and Notations}
We assume that we have a set of $M$ constraints forming the $M$ dimensional vector
\begin{equation}
\label{eq:C}
	\C(\q) = \vec{0}.
\end{equation}
The successive time derivatives must be zero as well, which produces the legal velocities
\begin{equation}
	\label{eq:J}
	\J\dq = \vec{0}
\end{equation}
where $\J$ is the Jacobian of $\C$ with respect to $\q$ and is a $M\times3N$ matrix.
We get the legal accelerations
\begin{equation}
	\dJ\dq + \J\ddq = \vec{0}
\end{equation}
\subsection{Virtual Forces}
Since the constraints are internal to the system, they doesn't work but they produce some internal
forces $\G$. So
\begin{equation}
\forall \dq, \; \G\cdot\dq = 0.
\end{equation}
Accordingly, $\G$ is a combination of the rows of $J$
\begin{equation}
	\G = \tJ \vec{\lambda}.
\end{equation}
Let us write
\begin{equation}
	\ddq = \W \left( \vec{F} + \G \right)
\end{equation}
we obtain
\begin{equation}
	\J \W \tJ \vec{\lambda} = -\left\lbrack \J\W\vec{F} + \dJ \dq \right\rbrack
\end{equation}
and
\begin{equation}
	\G = - \tJ \left(\J \W \tJ \right)^{-1}\left\lbrack \J\W\vec{F} + \dJ \dq \right\rbrack.
\end{equation}

\subsection{Algebraic Considerations}
The matrix
\begin{equation}
	 \J \W \tJ 
\end{equation}
is the Gram matrix of $\J\W^{1/2}$.
Provided that the rows of $\J$ are independent, namely $\text{rank}(\J)=M$, and likely that the $M$ constraints are independent, 
then this is a definite positive, and the Cholesky algorithm can be used to solve the linear problem.

\section{Discrete Newtonian Dynamics}

\subsection{Legalizing the phase space}
Let us assume that $\q_k$ and $\dq_{k}$ are almost satisfying respectively Eq. \eqref{eq:C} and Eq. \eqref{eq:J}.
To change the positions, one can move the particles only along $\W\tJ_k \vec{\lambda}_k$ to respect the weighted constraints physics.
We obtain some \textbf{upgraded positions}
\begin{equation}
	\label{eq:legal_q}
	\q_{k+1} = \q_k - \W \tJ_k \left(\J_k \W \tJ_k \right)^{-1} \C(\q_k)
\end{equation}
and we can compute $\J_{k+1}$.
We the update $\dq_k$ to some \textbf{legal velocities}
\begin{equation}
	\label{eq:legal_dq}
	\dq_{k+1} = \dq_k - \W\tJ_{k+1} \left(\J_{k+1} \W \tJ_{k+1} \right)^{-1} \J_{k+1} \dq_k.
\end{equation}
We repeat this until the desired tolerance on $\q_{k+1}$ is achieved.
At the end of the loop, a valid input phase space $\q_{n}$, $\dq_{n}$ and $\J_n$ is available.

\subsection{Constrained Velocity Verlet}

Starting from a valid $\q_n$, we compute $\J_n$, and we must also have a valid $\dq_n$ such that $\J_n\dq_n=\vec{0}$.

\begin{enumerate}
	\item Evaluate $\vec{F}_n = \vec{F}(\q_n,\dq_n)$ and $\dJ_n$.
	\item Evaluate $\G_n = -\tJ_n \left(\J_n \W \tJ_n \right)^{-1}\left\lbrack \J_n\W\vec{F}_n + \dJ_n \dq_n \right\rbrack$
	\item Evaluate the predicted phase space
	\begin{align*}
	\dq_{n+\half}^p & = \dq_n + \frac{1}{2} \tau \W \left(\vec{F}_n + \G_n \right) \\
	\q_{n+\half}^p  & = \q_{n} + \tau \dq_{n+\half}^p
	\end{align*}
	\item Legalize $\left(\q_{n+\half}^p,\dq_{n+\half}^p\right)$ to $\left(\q_{n+1},\dq_{n+\half}\right)$. Now $\J_{n+1}$ is available.
	\item Evaluate $\vec{F}_{n+\half}=\vec{F}\left(\q_{n+1},\dq_{n+\half}\right)$, $\dJ_{n+\half}$, 
	and 
	$$
	\G_{n+\half}=-\tJ_{n+1} \left(\J_{n+1} \W \tJ_{n+1} \right)^{-1}\left\lbrack \J_{n+1}\W\vec{F}_{n+\half} + \dJ_{n+\half} \dq_{n+\half} \right\rbrack
	$$
	\item Update
	$$
		\dq_{n+1}^{p} = \dq_{n+\half} + \frac{1}{2} \tau \W\left(\vec{F}_{n+\half} + \vec{G}_{n+\half}\right)
	$$
	\item Legalize $\dq_{n+1}^{p}$ to $\dq_{n+1}$ as shown in Eq. \eqref{eq:legal_dq}
\end{enumerate}

\section{Practical Implementation}

\subsection{Notations}
Every time a matrix $\J_n$ is computed then  a representation of $\A_n = \left(\J_n \W\tJ_n\right)^{-1}$
must be computed as well.\\

An upgrade of positions is
$$
	\q_{k+1} = \q_{k} - \W \tJ_k  \A_{k} \C\left(\q_{k}\right).
$$

A legalization of the velocities is
$$
	\dq_{k+1} = \dq_{k} - \W \tJ_{k+1}  \A_{k+1} \J_{k+1} \dq_{k}
$$

Since $\W$ is a diagonal matrix, some dedicated multiplications shall take place.

\subsection{Algorithm Implementation}

\subsubsection{Initialization}
We assume that we start from a valid phase space $\left(\q_n,\dq_n\right)$.
We compute $\J_n$, $\A_n$ and collect $\vec{F}_n = \vec{F}\left(\q_n,\dq_n\right)$
The initial kinetic energy, potential energy and Virial are computed.

\subsubsection{Predicted Half Step}
\label{pred0}
Compute $\dJ_n$.
Then deduce $$\vec{F}^c_n=\vec{F}_n - \tJ_n\A_n\left(\J_n\W\vec{F}_n + \dJ_n\dq_n\right)$$

Use the predicted Velocity Verlet update with
$$
	\dq^p_{n+\half} = \dq_{n} + \frac{1}{2} \tau \W \vec{F}^c_{n}
$$
then
$$
	\q^p_{n+\half} = \q_{n} + \tau \dq^p_{n+\half}
$$

\subsubsection{Legalized Half-Step}
After at least one round, we have the matrix $\J_{n+1}$ and $\A_{n+1}$, and the
valid phase space $\left(\q_{n+1},\dq_{n+\half}\right)$.

\subsubsection{Midpoint Forces Evaluation}
Evaluate $\dJ_{n+\half}$ then collect
$$
	\vec{F}_{n+\half} = \vec{F}\left(\q_{n+1},\dq_{n+\half}\right)
$$
and compute
$$
	\vec{F}^c_{n+\half}=\vec{F}_{n+\half} - \tJ_{n+1}\A_{n+1}\left(\J_{n+1}\W\vec{F}_{n+\half} + \dJ_{n+\half}\dq_{n+\half}\right)
$$

\subsubsection{Predicted Full Step}
Update
$$
	\dq^p_{n+1} = \dq^p_{n+\half} + \frac{1}{2} \tau \vec{F}^c_{n+\half}
$$

\subsubsection{Legalized Full Step}
\label{final}
This is a one step correction with
$$
	\dq_{n+1} = \dq_{n+1}^p - \W\tJ_{n+1} \A_{n+1} \J_{k+1} \dq_{n+1}^p.
$$
The compute
$$
	\vec{F}_{n+1} = \vec{F}\left(\q_{n+1},\dq_{n+1}\right)
$$

To go further, iterate to step \ref{pred0}.

\subsection{Comments}

\subsubsection{Constraints Clustering}
Obviously, the constraints are most of the time grouped into clusters contribute
independently to their involved particles. This shall greatly reduce the dimensionality of the matrices.

\subsubsection{Holonomic Forces}
When the forces don't depend on the velocities, then the forces evaluation in step \ref{final} is unnecessary.


\end{document}

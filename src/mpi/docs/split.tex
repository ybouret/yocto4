\documentclass[aps]{revtex4}
\usepackage{graphicx}
\usepackage{amssymb,amsfonts,amsmath}
\usepackage{chemarr}
\usepackage{bm}
\usepackage{pslatex}
%\usepackage{mathptmx}
\usepackage{xfrac}

\begin{document}
\title{MPI regular splitting}
\maketitle
	
\textbf{Description and notations.}
Let's say we have a 2D box with lengths $L_x,L_x$ and
we want to split the layout with $N$ cores, under the assumption
of fully periodic boundary conditions.
The time scale is the communication time across a domain.
The average work time per grid point is $\alpha$, and
a local copy time is $\beta$. The raw
work time is $\Theta_N=\alpha\dfrac{L_xL_y}{N}$. We need
the communication times to compare to $T_0=\alpha L_xL_y$.\\

\textbf{$X$-splitting.}
The communication time is $\theta_x=2\left(L_y + \beta \dfrac{L_x}{N}\right)$.\\

\textbf{$Y$-splitting.}
The communication time is $\theta_y=2\left(L_x + \beta \dfrac{L_y}{N}\right)$.\\

We see that $\theta_x-\theta_y=2\left(1-\dfrac{\beta}{N}\right)\left(L_y-L_x\right)$: assuming that $\beta<1$, we
shall cut in the larger direction.\\

\textbf{$XY$-splitting.} We assume we set $n_x,n_y$ patches so that $n_x+n_y=N$,
with $n_x>1$ and $n_y>1$, otherwise we are back to $X$- or $Y$-splitting.\\
The communication time is about $\tau_{xy}=2\left(\dfrac{L_y}{n_y}+\dfrac{L_x}{n_x}\right)$.
\end{document}

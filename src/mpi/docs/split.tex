\documentclass[aps]{revtex4}
\usepackage{graphicx}
\usepackage{amssymb,amsfonts,amsmath}
\usepackage{chemarr}
\usepackage{bm}
\usepackage{pslatex}
%\usepackage{mathptmx}
\usepackage{xfrac}

\begin{document}
\title{Regular splitting of quad meshes}
\maketitle
We have a 2D box with lengths $L_x,L_x$ and
we want to split the layout with $N$ regions, under the assumption
of fully periodic boundary conditions.\\
 The computation
that we want to carry out depends on both the size of
a region and the time to transfer information between two
regions.\\
We assume that $\alpha$ is the time to transfer the needed data 
between two regions \emph{per unit of length}, and
we neglect the local copy times.
The sequential compute time is $\tau$.
\begin{itemize}
\item If we split along $X$, we have a compute time of
\begin{equation}
	\Theta_x = \dfrac{\tau}{N} + 4\alpha L_y,
\end{equation}
the factor $4$ arising from two send, two receive operations...

\item If we split along $Y$, we have a compute time of
\begin{equation}
	\Theta_y = \dfrac{\tau}{N} + 4\alpha L_x
\end{equation}
\end{itemize}

We remark that
\begin{equation}
	\Theta_x - \Theta_y = 4\alpha\left(L_y-L_x\right),
\end{equation}
so we would always split in the largest direction.\\


If we split along $XY$ with $N=n_x \times n_y$ and $n_x>1,n_y>1$ (otherwise we are linear),
we get the compute time of
\begin{equation}
	\Theta_{xy} = \dfrac{\tau}{N} + 4\alpha\left(\dfrac{L_x}{n_x}+\dfrac{L_y}{n_y}\right).
\end{equation}
Let us assume that, by symmetry, $L_x\geq L_y$, so that we want to evaluate
\begin{equation}
	\delta \Theta = \Theta_{xy}-\Theta_x = 4\alpha\left(\dfrac{L_x}{n_x}+\dfrac{L_y}{n_y}\right) -
	4\alpha L_y.
\end{equation}	
We get
\begin{align}
	\delta \Theta \leq 0 & \Leftrightarrow n_yL_x + n_xL_y - N L_y \leq 0 \\
	& \Leftrightarrow \dfrac{N}{n_x} L_x + n_x L_y - N L_y \leq 0\\
	& \Leftrightarrow N L_x + n_x^2 L_y - n_x N L_y \leq 0\\
	& \Leftrightarrow n_x^2 - N n_x + N \dfrac{L_x}{L_y} \leq 0
\end{align}

The discriminant of the left side is 
\begin{equation}
	\Delta = N^2 - 4 N \dfrac{L_x}{L_y}.
\end{equation}
\begin{itemize}
	\item If 
		\begin{equation}
		\dfrac{L_x}{L_y}\geq \dfrac{N}{4}
		\end{equation}
		ie the shape is too "long" in the $X$ direction, it's always better to split along $X$ only.
	\item Otherwise, it may be possible to choose $n_x$ between the two roots of the previous equation
	and if $n_x>1$, $n_y>1$ and $n_x \times n_y = N$.
\end{itemize}

\end{document}


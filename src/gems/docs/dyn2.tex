\documentclass[aps,twocolumn]{revtex4}
\usepackage{graphicx}
\usepackage{amssymb,amsfonts,amsmath}
\usepackage{chemarr}
\usepackage{bm}
\usepackage{pslatex}
%\usepackage{mathptmx}
\usepackage{xfrac}

\newcommand{\mymat}[1]{\boldsymbol{#1}}
\newcommand{\mytrn}[1]{{\!\!~^{\mathsf{t}}{#1}}}
\newcommand{\myvec}[1]{\overrightarrow{#1}}
\newcommand{\half}{\sfrac{1}{2}}
\newcommand{\onethird}{\sfrac{1}{3}}
\newcommand{\twothirds}{\sfrac{2}{3}}
\newcommand{\q}{\vec{q}}
\newcommand{\dq}{\dot{\q}}
\newcommand{\ddq}{\ddot{\q}}
\newcommand{\C}{\vec{C}}
\newcommand{\J}{\mymat{J}}
\newcommand{\dJ}{\dot{\J}}
\newcommand{\tJ}{\mytrn{\J}}
\newcommand{\G}{\vec{G}}
\newcommand{\W}{\mymat{W}}
\newcommand{\A}{\mymat{A}}
\newcommand{\JW}{\hat{\mymat{J}}}
\newcommand{\tJW}{\mytrn{\JW}}

\begin{document}

\title{Some particles dynamics}
\maketitle

\section{Setup}

Let us assume that we have a set of $N$ particles with positions $\q$.
Those particles are submitted to some forces $\vec{F}$ such that
\begin{equation}
	\ddq = \W \vec{F}
\end{equation}

\section{Velocity Verlet}
We start from $\q_n$, $\dq_n$ and $\ddq_n         = \W \vec{F}\left(\q_n,\dq_n\right)$.

\begin{enumerate}
\item Evaluate $\dq_{n+\half}  = \dq_n + \dfrac{1}{2}\tau \ddq_n$
\item Evaluate $\q_{n+1} = \q_{n} + \tau \dq_{n+\half}$
\item Evaluate $\ddq_{n+\half} = \W \vec{F}\left(\q_{n+1},\dq_{n+\half}\right)$
\item Update   $\dq_{n+1}      = \dq_{n+\half} + \dfrac{1}{2} \tau \ddq_{n+\half}$
\item Evaluate $\ddq_{n+1}     = \W \vec{F}\left(\q_{n+1},\dq_{n+1}\right)$
\end{enumerate}

\section{Adding constraints}
We have to deal with a set of $M$ constraints 
\begin{equation}
	\C\left(\q\right) = \vec{0}.
\end{equation}
This also defines the legal velocities since
\begin{equation}
\dfrac{d \C}{dt} = \vec{0} = \J \dq
\end{equation}
where $\J$ is the Jacobian of $\C$ w.r.t. $\q$.
Let us assume that we start from $\q_n$ and $\dq_n$ that match the two previous relations.
Since we have a discrete algorithm, we assume that
an extra force is acting along a combination of $\mytrn{\J}$.

We can modify the Velocity Verlet algorithm.

We start from $\q_n$, $\dq_n$ and $\ddq_n = \W \vec{F}\left(\q_n,\dq_n\right)$,
assuming $\C\left(\q_n\right) = \vec{0}$ and $\J_n \dq_n =\vec{0}$.
\begin{enumerate}
\item We compute $\q_{n+\half} = \q_{n} + \tau \dq_n + \dfrac{1}{2} \tau^2 \ddq_n$.
\item We solve for $\vec{\lambda}$ such that
\begin{equation}
	\C\left(\q_{n+\half}+\W\mytrn{\J}_n\vec{\lambda}\right) = \vec{0} 
\end{equation}
using ONLY $\J_n$ as the Jacobian of $\C$.

\item We compute $\delta\q_n=\W\mytrn{\J}_n\vec{\lambda}$ and $\q_{n+1} = \q_{n+\half} + \delta\q_n$.
\item We evaluate $\J_{n+1}$.
\item We evaluate the half step velocity increase
	$$
		\dq_{n+\onethird} = \dq_n + \dfrac{1}{2} \tau \ddq_n + \dfrac{1}{\tau} \delta\q_n
	$$
\item We compute $\dq_{n+\twothirds}$ such that
$$
	\J_{n+1} \dq_{n+\twothirds} = \vec{0}
$$
\end{enumerate}

\end{document}


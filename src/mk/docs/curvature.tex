\documentclass[aps,onecolumn]{revtex4}
\usepackage{graphicx}
\usepackage{amssymb,amsfonts,amsmath,amsthm}
\usepackage{chemarr}
\usepackage{bm}
\usepackage{pslatex}
\usepackage{mathptmx}
\usepackage{xfrac}
\usepackage{xcolor}

\begin{document}

\title{Curvature of a surface in $\mathbb{R}^3$}
\maketitle

Let $\mathcal{S}$ be a surface in $\mathbb{R}^3$ define by
\begin{equation}
	\mathcal{S} = \lbrace \vec{Q}(u,v) \vert  (u,v) \in \mathbb{R}^2 \rbrace.
\end{equation}
We define 
\begin{itemize}
\item the two tangent vectors
\begin{equation}
	\vec{Q}_u = \partial_u \vec{Q},\;\;\vec{Q}_v = \partial_v \vec{Q}
\end{equation}

\item the normal direction
	\begin{equation}
		\vec{n} = \vec{Q}_u \wedge \vec{Q}_v
	\end{equation}	

\item the three second tangent vectors
\begin{equation}
	\vec{Q}_{uu} = \partial_{uu} \vec{Q},\;\;\vec{Q}_{vv}=\partial_{vv} \vec{Q},\;\;
	\vec{Q}_{uv} = \vec{Q}_{vu} = \partial_{uv} \vec{Q} = \partial_{vu} \vec{Q}.
\end{equation}

\end{itemize}

We then compute the quantities
\begin{equation}
	E = \vec{Q}_u^2, \;\; F = \vec{Q}_u \cdot \vec{Q}_v,\;\; G = \vec{Q}_v^2
\end{equation}
and
\begin{equation}
	L = \vec{n} \cdot \vec{Q}_{uu},\;\; M =\vec{n} \cdot \vec{Q}_{uv}, \;\; N = \vec{n}\cdot\vec{Q}_{vv}.
\end{equation}
The average curvature is then
\begin{equation}
	H = \dfrac{1}{2\left\vert\vec{n}\right\vert} \dfrac{EN+GL-2FM}{EG-F^2}
\end{equation}
and the gaussian curvature is
\begin{equation}
	K = \dfrac{1}{\vec{n}^2}\dfrac{LN-M^2}{EG-F^2}
\end{equation}

\end{document}
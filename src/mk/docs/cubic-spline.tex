\documentclass[aps]{revtex4}
\usepackage{graphicx}
\usepackage{amssymb,amsfonts,amsmath,amsthm}
\usepackage{chemarr}
\usepackage{bm}
\usepackage{pslatex}
\usepackage{mathptmx}
\usepackage{xfrac}

%% concentration notations
\newcommand{\mymat}[1]{\boldsymbol{#1}}
\newcommand{\mytrn}[1]{{#1}^{\mathsf{T}}}
\newcommand{\myvec}[1]{\overrightarrow{#1}}
\newcommand{\mygrad}{\vec{\nabla}}
\newcommand{\myhess}{\mathcal{H}}


\begin{document}
\title{Cubic B-Splines}
\maketitle

Let $P_0$, $P_1$, $P_2$ and $P_3$ be control points at $u_0$, $u_1$, $u_2$ and $u_3$.
The cubic B-spline $S(x=u-u_0)$ is defined by
$$
	S(x) = P_0 + \dfrac{P_1-P_0}{u_1-u_0} x + b x^2 + c x^3.
$$
so that
$$
	(P_3-P_0) - x_3 \dfrac{P_1-P_0}{u_1-u_0} = b x_3^2 + c x_3 ^3.
$$
Since
$$
	S'(x) = \dfrac{P_1-P_0}{u_1-u_0} + 2 b x + 3 c x^2
$$
the other equation is
$$
	\frac{P_3-P_2}{u_3-u_2} - \dfrac{P_1-P_0}{u_1-u_0} = 2 b x_3 + 3 c x_3^2
$$
Using
$$
	D_0 = \dfrac{P_1-P_0}{u_1-u_0}
$$
and
$$
 D_3 = \frac{P_3-P_2}{u_3-u_2}
$$	
we obtain
$$
	S(x) = P_0 + D_0 x - \left[\frac{(D_3+2D_0)x_3 + 3(P_0-P_3)}{x_3^2}\right] x^2
	+ \left[ \frac{(D_0+D_3)x_3 + 2(P_0-P_3)}{x_3^3}\right] x^3
$$
\end{document}
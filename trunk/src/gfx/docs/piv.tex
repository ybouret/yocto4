\documentclass[aps,twocolumn]{revtex4}
\usepackage{graphicx}
\usepackage{amssymb,amsfonts,amsmath,amsthm}
\usepackage{chemarr}
\usepackage{bm}
\usepackage{pslatex}
\usepackage{mathptmx}
\usepackage{xfrac}


\begin{document}
\title{P.I.V.}

\section{The Data}
We have two frames $F_0$ and $F_1$ with some tracers following the
velocity field. How to recompose the apparent 2D velocity field ?

The main problems are that in fact the field is 3D, and that
on the sides some particles may appear or disappear...

\section{Direct Field Reconstruction}
Let us assume that we can follow a certain quantity $\rho$ which is purely advected by
the field $\vec{V}$.
Then the advection equation is
$$
\partial_t \rho + \mathrm{div}\left(\rho\vec{V}\right) = \sigma
$$

\subsection{1D}
Let us assume that in the bulk we have no extra source term.
The equation becomes
$$
	\partial_t \rho + \rho \partial_x V + V \partial_x \rho = 0
$$
or using $p=V,q=\partial_x V,g=\partial_x\rho$,
$$
	\rho(t+dt) - \rho(t) \propto  \rho q + p g
$$
so that we want to minimise
$$
	\mathcal{L}(p,q) = \dfrac{1}{2}\int (\rho q + p g)^2 d\vec{r}.
$$
The Euler-Lagrange condition writes
$$
	\dfrac{\partial \mathcal{L}}{\partial p} -
	\dfrac{\partial}{\partial x} \left( \dfrac{\partial \mathcal{L}}{\partial q}\right) = 0
$$
\end{document}
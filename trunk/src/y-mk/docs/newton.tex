\documentclass[aps,twocolumn]{revtex4}
\usepackage{graphicx}
\usepackage{amssymb,amsfonts,amsmath,amsthm}
\usepackage{chemarr}
\usepackage{bm}
\usepackage{pslatex}
\usepackage{mathptmx}
\usepackage{xfrac}

%% concentration notations
\newcommand{\mymat}[1]{\boldsymbol{#1}}
\newcommand{\mytrn}[1]{{#1}^{\mathsf{T}}}
\newcommand{\myvec}[1]{\overrightarrow{#1}}
\newcommand{\mygrad}{\vec{\nabla}}
\newcommand{\myhess}{\mathcal{H}}

\begin{document}

\section{Setup}
Let $\vec{F}\left(\vec{X}\right)$ be a system of $n$ equations with $n$ variables.
Starting from $\vec{X}_k$, what is the step to take ?
In any case, for a small step $\vec{h}$ we have
\begin{equation}
	\vec{F}\left(\vec{X}_k+\vec{h}\right) = \vec{F}_k + \mymat{J} \vec{h}
\end{equation}
We define the associated objective function
\begin{equation}
	G\left(\vec{X}\right) = \frac{1}{2} \vec{F}^2
\end{equation}
We notice that
\begin{equation}
	\mygrad G = \mytrn{\mymat{J}}\vec{F}
\end{equation}
and that
\begin{equation}
	\myhess_G = \mytrn{\mymat{J}}\mymat{J} + \partial_{\vec{X}}\mymat{J} \otimes \vec{F}
\end{equation}

\section{Well conditioned Jacobian}
In that case, we can compute the full Newton's step
\begin{equation}
	\vec{h}_k = - \mymat{J}^{-1} \vec{F}_k
\end{equation}

\end{document}
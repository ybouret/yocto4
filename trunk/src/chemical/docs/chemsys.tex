\documentclass[aps,twocolumn]{revtex4}
\usepackage{graphicx}
\usepackage{amssymb,amsfonts,amsmath,amsthm}
\usepackage{chemarr}
\usepackage{bm}
\usepackage{pslatex}
\usepackage{mathptmx}
\usepackage{xfrac}

%% concentration notations
\newcommand{\myconc}[1]{\left\lbrack #1 \right\rbrack}
\newcommand{\mychem}[1]{{\mathtt{#1}}}
\newcommand{\species}{\mychem{X}}
\newcommand{\mymat}[1]{\boldsymbol{#1}}
%\newcommand{\mytrn}[1]{{#1}^{\mathrm{t}}}
\newcommand{\mytrn}[1]{{\!\!~^{\mathsf{t}}{#1}}}
\newcommand{\myvec}[1]{\overrightarrow{#1}}
\newcommand{\vecX}{\myvec{\myconc{\mychem{X}}}}

\begin{document}
\title{Chemical Solution}


\maketitle
%\tableofcontents

\section{Chemical system Equilibria}
\subsection{Formal Description}
If we assume that we have $N$ chemical reactions coupling $M$ species $\species_{1,\ldots,M}$ then those reactions can be classically written with the help
of the \textit{algebraic} stoichiometric coefficients matrix $\mymat{\nu}$ as
\begin{equation}
	\label{eq:Ki}
	\forall i \in 1..N, \; \sum_{j\in1..M} \mymat{\nu}_{i,j} \species_j = 0, \; K_i=\prod_{j\in1..M} \myconc{\species_j}^{\mymat{\nu}_{i,j}}
\end{equation}
where $K_i$ is the equilibrium constant of the i$^{\text{th}}$ reaction\\
Accordingly we can define the vector $\vec{\Gamma}$ such that its coordinate $\vec{\Gamma}_i$
\begin{equation}
	\label{eq:Gamma}
	\forall i \in 1..N, \vec{\Gamma}_i = K_i \prod_{\mymat{\nu}_{i,j}<0} \myconc{\species_j}^{-\nu_{i,j}} - \prod_{\mymat{\nu}_{i,j}>0} \myconc{\species_j}^{\nu_{i,j}} = 0.
\end{equation}
The Jacobian matrix of $\vec{\Gamma}$ is
\begin{equation}
	\partial_{\vecX} \vec{\Gamma} = \mymat{\Phi}
\end{equation}
The $N$ equilibria allow a vector of corresponding chemical extent $\delta\vec{\xi}$ such that any increase of concentrations is
\begin{equation}
	\delta\vecX = \mytrn{\mymat{\nu}} \delta\vec{\xi}
\end{equation}

\subsection{Valid Equilibria: Newton Algorithm Mark-I}
\subsubsection{Algebraic Method}

Let us assume that we have a set of concentration $\vecX_{n}$ and that we look for an evolution $\delta\vec{\xi}_{n}$.
\begin{equation}
	\vec{\Gamma}\left(\vecX_{n}+\delta\vecX_{n}\right) \simeq \vec{\Gamma}_{n} + \mymat{\Phi}_n \mytrn{\mymat{\nu}} \delta\vec{\xi}_n
\end{equation}
For a valid set of concentrations (all greater of equal to zero, one not zero), the matrix $\mymat{\Phi}_n \mytrn{\mymat{\nu}}$ is invertible
and
\begin{equation}
	\delta\vec{\xi}_n = -\left(\mymat{\Phi}_n \mytrn{\mymat{\nu}}\right)^{-1} \vec{\Gamma}_{n}
\end{equation}
so that the concentration increase is
\begin{equation}
	\delta\vecX_{n} = - \mytrn{\mymat{\nu}}\left(\mymat{\Phi}_n \mytrn{\mymat{\nu}}\right)^{-1} \vec{\Gamma}_{n}
\end{equation}
and
\begin{equation}
	\vecX_{n+1} = \vecX_{n} + \delta\vecX_{n}.
\end{equation}
This Newton step is to be repeated until
\begin{equation}
	\forall i, \; \vert\delta\vecX_{n,i}\vert \leq \epsilon \times \vert\vecX_{n+1,i}\vert
\end{equation}

\subsubsection{Numerical Problems/Cutoffs}
We have to control the small concentrations and the small increase of concentration.
Let us assume that $\tilde\epsilon$ is near the numerical zero. Then any concentration under this value
should be set to zero. 


\section{Chemical System Initialization}
\subsection{Remaining Degrees of Freedom}
We need $M-N$ supplementary relations to initialize the system. Hopefully, this constraint should be linear
since they shall express the matter conservation, the charge conservation or a fixed concentration.
Let us assume that $\mymat{P}\in\mathcal{M}_{M-N,M}$ is the corresponding "projection" matrix such that
\begin{equation}
	\label{eq:proj}
	\mymat{P} \vecX = \vec{\Lambda} \in \mathbb{R}^{M-N}.
\end{equation}

\subsection{Finding a matching solution: the virtual source method}
\subsubsection{Starting point}
Let us start from a valid composition: a \textbf{random scaled composition} is computed then the Newton's algorithm is
applied to find $\vecX_0$ such that :
$$\vec{\Gamma}(\vecX_0)=\vec{0}.$$

\subsubsection{Virtual source term}
%We know that there exist two vectors $\vec{U}\in\mathbb{R}^{M-N}$ and $\vec{V}\in \mathbb{R}^{N}$ of Lagrange multipliers, and
%a matrix  $\mymat{Q}\in\mathcal{M}_{N,M}$ such that $\mymat{Q}\in\mymat{P}^\perp$
%\begin{equation}
%	\vecX = \mytrn{\mymat{P}} \vec{U} + \mytrn{\mymat{Q}}\vec{V}.
%\end{equation}
Starting from $\vecX_n$, we want to compute $\vec{\sigma}_n$  such that
$$
	\mymat{P}(\vecX_n+\vec{\sigma}_n) = \vec{\Lambda}
$$
or
$$
	\mymat{P} \vec{\sigma}_n = \vec{\Lambda} - \mymat{P} \vecX_n.
$$
This can be achieved if $\vec{\sigma}_n$ is in the image of $\mymat{P}$, so that there exist a vector $\vec{U}_n\in\mathbb{R}^{M-N}$ 
such that
$$
	\vec{\sigma}_n = \mytrn{\mymat{P}} \vec{U}_n
$$
We recognise an invertible Gram matrix so that finally
\begin{equation}
		\vec{\sigma}_n = \underbrace{\mytrn{\mymat{P}} \left(\mymat{P} \mytrn{\mymat{P}}\right)^{-1}}_{\text{Moore-Penrose Pseudo Inverse}}
		 \left( \vec{\Lambda} - \mymat{P}\vecX_n \right)
\end{equation}

\subsubsection{Chemical Damping}
The source term must be damped by the equilibria, so that we have
\begin{equation}
	\vec{\rho}_n = \left[\mymat{I} - \mytrn{\mymat{\nu}}\left(\mymat{\Phi}_n\mytrn{\mymat{\nu}}\right)^{-1}\mymat{\Phi}_n\right] \vec{\sigma_n}
\end{equation}
and we build
\begin{equation}
	\vec{Y}_n = \vecX_n + \vec{\rho}_n \text{ and } \forall i, Y_i \geq 0
\end{equation}
and $\vecX_{n+1}$ is the Newton's algorithm solution starting from $\vec{Y}_n$.
\end{document}



\subsubsection{Linear Improvement}
Due to numerical roundoff, $\vecX_n$ maybe not numerically legal.
We iterate the projection
\begin{equation}
	\left\lbrace
	\begin{array}{rcl}
	\delta \vecX_n & = & - \mytrn{\mymat{P}} \left(\mymat{P} \mytrn{\mymat{P}}\right)^{-1} \left( \mymat{P}\vecX_n - \vec{\Lambda} \right)\\
	\vecX_{n+1}    & = & \vecX_n + \delta\vecX_n
	\end{array}
	\right.
\end{equation}
until the numerical limit is reached.

\subsubsection{Final Error Estimation}
Let $\vecX_n$ the set of current values. The the error is about
\begin{equation}
	\delta \vecX_n = \mytrn{\mymat{Q}}\left(\mymat{\Phi}_n\mytrn{\mymat{Q}}\right)^{-1} \vec{\Gamma}_n
\end{equation}
This shall provide the cutoff for small concentrations.


\end{document}

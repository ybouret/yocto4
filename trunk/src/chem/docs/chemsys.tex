\documentclass[aps]{revtex4}
\usepackage{graphicx}
\usepackage{amssymb,amsfonts,amsmath,amsthm}
\usepackage{chemarr}
\usepackage{bm}
\usepackage{bbm}
\usepackage{pslatex}
\usepackage{mathptmx}
\usepackage{xfrac}

\newcommand{\mymat}[1]{\bm{#1}}
\newcommand{\mytrn}[1]{~^{\mathsf{T}}{#1}}
\newcommand{\mygrad}{\vec{\nabla}}

\begin{document}
\title{Chemical Solutions}

\section{Description}
Let us assume that we have $A_1,\ldots,A_M$ chemical species coupled by
$N$ equilibria such that
\begin{equation}
	\forall i \in [1;N], \;\; \sum_{j=1}^{M} \nu_{i,j} C_i = 0, \;\; K_i(t) = \prod_{i=1}^{M} C_i^{\nu_{i,j}}.
\end{equation}
We remove the singularities by assuming that the equilibria are met when
\begin{equation}
	\forall i \in [1;N], \;\; \Gamma_i(t,\vec{C}) = K_i(t) \prod_{\nu_{i,j}<0}  C_i^{-\nu_{i,j}} -  \prod_{\nu_{i,j}>0} C_i^{\nu_{i,j}} 
\end{equation}
or
\begin{equation}
	\vec{\Gamma}(t,\vec{C}) = \vec{0}.
\end{equation}
We also naturally have the topology matrix $\mymat{\nu}$.

\section{Acceptable Extent}
If an extent is to be applied to a set of concentrations, we must have
a final positive concentration of \textbf{active} species (namely implied in one or more reactions).
We use the function
$$
	E(C) = 
	\left\lbrace
	\begin{array}{rcl}
	-\frac{1}{2} C^2 & \text{if} & C < 0\\
	0 & \text{if} & C \geq 0\\
	\end{array}
	\right.
$$
and the objective function
$$
	\mathcal{E}(\vec{C}) = \sum_{j\in\text{active species}} E(C_j).
$$
Starting from a possibly invalid concentration $\vec{C}_k$, 
we want to minimise (even set to zero) $\mathcal{E}$ by using a chemical transformation $\vec{\xi}_k$, so 
that $\vec{\xi}_k$ minimises 
$$
	\mathcal{E}(\vec{C}_k + \mytrn{\mymat{\nu}}\vec{\xi}).
$$
The gradient of this function is
$$
	\mygrad_{\vec{\xi}} \mathcal{E} = \mymat{\nu} \cdot
	\begin{bmatrix}
	-C_1 \cdot \mathbbm{1}_{C_1<0} \\
	\vdots\\
	-C_M \cdot \mathbbm{1}_{C_M<0}\\
	\end{bmatrix}.
$$
The structure of $\mymat{\nu}$ allows the writing of the right vector without taking care of active/inactive species.
A simple steepest descent algorithm shall find $u$ such that
$$
	\vec{C}_{k+1} = \vec{C}_k - u \mytrn{\mymat{\nu}} \mymat{\nu} \begin{bmatrix}
	-C_1 \cdot \mathbbm{1}_{C_1<0} \\
	\vdots\\
	-C_M \cdot \mathbbm{1}_{C_M<0}\\
	\end{bmatrix} =
	\vec{C}_k + u \mytrn{\mymat{\nu}} \mymat{\nu} \begin{bmatrix}
	C_1 \cdot \mathbbm{1}_{C_1<0} \\
	\vdots\\
	C_M \cdot \mathbbm{1}_{C_M<0}\\
	\end{bmatrix}
	.
$$

\section{Finding some equilibria}

Let us assume that we start from a valid but out-of-equilibrium state $\vec{C}$.
Then the system can evolve only by a $N$ dimensional chemical advancement $\vec{\xi}$
\begin{equation}
	\vec{C}_{eq} = \vec{C} + \mytrn{\mymat{\nu}}\vec{\xi}
\end{equation}
Starting from a value $\vec{C}_{k}$, the Newton algorithm yields
\begin{equation}
	\vec{0} = \vec{\Gamma}(t,\vec{C}_k) + \mymat{\Phi}(t,\vec{C}_k) \mytrn{\mymat{\nu}}\vec{\xi}_k
\end{equation}

\end{document}
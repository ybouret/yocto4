\documentclass[aps]{revtex4}
\usepackage{graphicx}
\usepackage{amssymb,amsfonts,amsmath,amsthm}
\usepackage{chemarr}
\usepackage{bm}
\usepackage{bbm}
\usepackage{pslatex}
\usepackage{mathptmx}
\usepackage{xfrac}

\newcommand{\mymat}[1]{\bm{#1}}
\newcommand{\mytrn}[1]{~^{\mathsf{T}}{#1}}
\newcommand{\mygrad}{\vec{\nabla}}

\begin{document}
\title{Chemical Solutions}

\section{Description}
Let us assume that we have $A_1,\ldots,A_M$ chemical species coupled by
$N$ equilibria such that
\begin{equation}
	\forall i \in [1;N], \;\; \sum_{j=1}^{M} \nu_{i,j} C_i = 0, \;\; K_i(t) = \prod_{i=1}^{M} C_i^{\nu_{i,j}}.
\end{equation}
We remove the singularities by assuming that the equilibria are met when
\begin{equation}
	\forall i \in [1;N], \;\; \Gamma_i(t,\vec{C}) = K_i(t) \prod_{\nu_{i,j}<0}  C_i^{-\nu_{i,j}} -  \prod_{\nu_{i,j}>0} C_i^{\nu_{i,j}} 
\end{equation}
or
\begin{equation}
	\vec{\Gamma}(t,\vec{C}) = \vec{0}.
\end{equation}
We also naturally have the topology matrix $\mymat{\nu}$.

\section{Acceptable Extent}
If an extent is to be applied to a set of concentration, we must have
$$
	\forall i,\;\forall j,\;C_i + \nu_{i,j}\xi_i \geq 0
$$
which define a set of acceptable extents, which may present a minimum, a maximum or both values.
It has to be computed.

\section{Acceptable Compositions}
Normally, each concentration must be positive. If not, then this should be corrected.

\subsection{Correcting by chemical extent}
We define the $\mathcal{C}^1$ function
$$
	E(x) = \left\lbrace
	\begin{array}{rcl}
	-\frac{1}{2} x^2 & \text{if} & x<0\\
	0                & \text{if} & x\geq0\\
	\end{array}
	\right.
$$
and the total function
$$
	\mathcal{E}(\vec{C}) = \sum_{i=1}^M E(C_i)
$$
which is globally minimised for any acceptable set of concentrations.
Starting from a concentration $\vec{C}_k$, we want to find a chemical extent $\vec{\xi}_k$ such that
$\vec{C}_k + \mytrn{\mymat{\nu}}\vec{\xi}_k$ goes towards an acceptable set.
This may be done by finding a minimum of $\mathcal{E}$.
We have the $N$ dimensional gradient
$$
	\mygrad_{\vec{\xi}} \mathcal{E} = \mymat{\nu} \mygrad_{\vec{C}} \mathcal{E}.
$$
In a steepest descent algorithm, we should look at a minimum of $\mathcal{E}$
by finding $u$ such that
$$
	\vec{C}_{k+1} = \vec{C}_{k} - u \mytrn{\mymat{\nu}} \mymat{\nu} \mygrad_{\vec{C}} \mathcal{E}.
$$
Of course, the $M\times M$ matrix $\mytrn{\mymat{\nu}} \mymat{\nu}$ has a rank $N$, meaning that
only $N$ independent combinations of concentrations may be corrected.
We obtain
$$
	\vec{C}_{k+1} = \vec{C}_{k} + u \mytrn{\mymat{\nu}} \mymat{\nu}
	\begin{bmatrix}
	C_1 \cdot \mathbbm{1}_{C_1<0}\\
	\vdots\\
	C_M \cdot \mathbbm{1}_{C_M<0}\\
	\end{bmatrix} 
$$
\section{Finding some equilibria}

Let us assume that we start from a valid but out-of-equilibrium state $\vec{C}$.
Then the system can evolve only by a $N$ dimensional chemical advancement $\vec{\xi}$
\begin{equation}
	\vec{C}_{eq} = \vec{C} + \mytrn{\mymat{\nu}}\vec{\xi}
\end{equation}
Starting from a value $\vec{C}_{k}$, the Newton algorithm yields
\begin{equation}
	\vec{0} = \vec{\Gamma}(t,\vec{C}_k) + \mymat{\Phi}(t,\vec{C}_k) \mytrn{\mymat{\nu}}\vec{\xi}_k
\end{equation}

\end{document}
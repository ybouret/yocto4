\documentclass[aps,twocolumn]{revtex4}
\usepackage{graphicx}
\usepackage{amssymb,amsfonts,amsmath}
\usepackage{chemarr}
\usepackage{bm}
\usepackage{pslatex}
%\usepackage{mathptmx}
\usepackage{xfrac}

%% concentration notations
\newcommand{\mymat}[1]{\boldsymbol{#1}}
%\newcommand{\mytrn}[1]{{#1}^{\mathrm{t}}}
\newcommand{\mytrn}[1]{{\!\!~^{\mathsf{t}}{#1}}}
\newcommand{\myvec}[1]{\overrightarrow{#1}}
\newcommand{\half}{\sfrac{1}{2}}
\newcommand{\q}{\vec{q}}
\newcommand{\dq}{\dot{\q}}
\newcommand{\ddq}{\ddot{\q}}
\newcommand{\C}{\vec{C}}
\newcommand{\J}{\mymat{J}}
\newcommand{\dJ}{\dot{\J}}
\newcommand{\tJ}{\mytrn{\J}}
\newcommand{\G}{\vec{G}}
\newcommand{\W}{\mymat{W}}

\begin{document}
\title{Dynamics}

\section{Notations}
Let us have a system with $N$ particles, so that we have a $3N$-dimensional
vector $\q$ of positions and a vector $\dq$ of velocities.
We assume that we can compute the force field $\vec{F}$ such that
\begin{equation}
	\label{eq:newton}
	\ddq = \mymat{W} \vec{F}
\end{equation}
where $\mymat{W}$ is the tensorized mass inverse matrix.

\section{Newtonian Dynamics}

\subsection{Problem}
At step $n$, we know $\q_n$, $\dq_n$ and $\ddq_n = \mymat{W} \vec{F}_n$.
How do we compute $\q_{n+1}$ and $\dq_{n+1}$after a time step $\tau$ ?

\subsection{Velocity Verlet Derivation for Holonomic Forces}
We update $\q_n$ by Taylor expansion during $\tau$ with
\begin{equation}
	\label{eq:forward}
	\q_{n+1} = \q_n + \tau \dq_n + \frac{1}{2}\tau^2 \ddq_n.
\end{equation}
Since we can evaluate $\vec{F}_{n+1}$ and $\ddq_{n+1}=\mymat{W}\vec{F}_{n+1}$, the backward Taylor expansion
gives
\begin{equation}
	\label{eq:backward}
	\q_n = \q_{n+1} - \tau \dq_{n+1} + \frac{1}{2}\tau^2 \ddq_{n+1}.
\end{equation}
The sum of Eq. \eqref{eq:forward} and Eq. \eqref{eq:backward} gives 
\begin{equation}
	\label{eq:vv}
	\dq_{n+1} = \dq_{n} + \dfrac{1}{2}\mymat{W}\left\lbrack\vec{F}_n + \vec{F}_{n+1}\right\rbrack
\end{equation}

\subsection{Velocity Verlet for Generic Forces}
We use the following sequence.
Given $\q_n$ and $\dq_n$, we compute $\vec{F}_n=\vec{F}(\q_n,\dq_n)$.
Then for each time step we perform the following sequence.
\begin{enumerate}
	\item Evaluate \begin{equation} 
	\dq_{n+\half} = \dq_n + \frac{1}{2}\tau \mymat{W}\vec{F}_n
	\end{equation}
	Advance
	\item \begin{equation}
	\q_{n+1}  = \q_n + \tau \dq_{n+\half}.
	\end{equation} 
	The potential energy is already available.
	\item Evaluate $\vec{F}_{n+\half}=\vec{F}(\q_{n+1},\dq_{n+\half})$
	\item Update
	\begin{equation}
		\dq_{n+1} = \dq_{n+\half} + \frac{1}{2} \tau \W \vec{F}_{n+\half}.
	\end{equation}
	The kinetic energy if available.
	\item Update
	\begin{equation}
		\vec{F}_{n+1}=\vec{F}(\q_{n+1},\dq_{n+1})
	\end{equation}
\end{enumerate}

\section{Holonomic Constraints}
\subsection{Definitions and Notations}
We assume that we have a set of $M$ constraints forming the $M$ dimensional vector
\begin{equation}
	\C(\q) = \vec{0}.
\end{equation}
The successive time derivatives must be zero as well, which produces the legal velocities
\begin{equation}
	\J\dq = \vec{0}
\end{equation}
where $\J$ is the Jacobian of $\C$ with respect to $\q$ and is a $M\times3N$ matrix.
We get the legal accelerations
\begin{equation}
	\dJ\dq + \J\ddq = \vec{0}
\end{equation}
\subsection{Virtual Forces}
Since the constraints are internal to the system, they doesn't work but they produce some internal
forces $\G$. So
\begin{equation}
\forall \dq, \; \G\cdot\dq = 0.
\end{equation}
Accordingly, $\G$ is a combination of the rows of $J$
\begin{equation}
	\G = \tJ \vec{\lambda}.
\end{equation}
Let us write
\begin{equation}
	\ddq = \W \left( \vec{F} + \G \right)
\end{equation}
we obtain
\begin{equation}
	\J \W \tJ \vec{\lambda} = -\left\lbrack \J\W\vec{F} + \dJ \dq \right\rbrack
\end{equation}
and
\begin{equation}
	\G = - \tJ \left(\J \W \tJ \right)^{-1}\left\lbrack \J\W\vec{F} + \dJ \dq \right\rbrack.
\end{equation}

\end{document}

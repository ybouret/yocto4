\documentclass[aps]{revtex4}
\usepackage{graphicx}
\usepackage{amssymb,amsfonts,amsmath,amsthm}
\usepackage{chemarr}
\usepackage{bm}
\usepackage{pslatex}
\usepackage{mathptmx}
\usepackage{xfrac}

%% concentration notations
\newcommand{\mymat}[1]{\boldsymbol{#1}}
\newcommand{\mytrn}[1]{{#1}^{\mathsf{T}}}
\newcommand{\myvec}[1]{\overrightarrow{#1}}
\newcommand{\mygrad}{\vec{\nabla}}
\newcommand{\myhess}{\mathcal{H}}


\begin{document}
\title{Cubic B-Splines}
\maketitle

Let $P_1$, $P_2$, $P_3$ and $P_4$ be control points at $u_1$, $u_2$, $u_3$ and $u_4$.
The cubic B-spline $S(x=u-u_1)$ is defined by
$$
	S(x) = P_1 + \dfrac{P_2-P_1}{u_2-u_1} x + b x^2 + c x^3.
$$
so that
$$
	(P_4-P_1) - x_4 \dfrac{P_2-P_1}{u_2-u_1} = b x_4^2 + c x_4 ^3.
$$
Since
$$
	S'(x) = \dfrac{P_2-P_1}{u_2-u_1} + 2 b x + 3 c x^2
$$
the other equation is
$$
	\frac{P_4-P_3}{u_4-u_3} - \dfrac{P_2-P_1}{u_2-u_1} = 2 b x_4 + 3 c x_4^2
$$
Using
$$
	D_1 = \dfrac{P_2-P_1}{u_2-u_1}
$$
and
$$
 D_4 = \frac{P_4-P_3}{u_4-u_3}
$$	
we obtain
$$
	S(x) = P_1 + D_1 x - \left[\frac{(D_4+2D_1)x_4 + 3(P_1-P_4)}{x_4^2}\right] x^2
	+ \left[ \frac{(D_1+D_4)x_4 + 2(P_1-P_4)}{x_4^3}\right] x^3
$$
\end{document}